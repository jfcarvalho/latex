% vim: tw=80 noai
\documentclass[normaltoc,capchap,capsec,times]{abnt}
\usepackage[latin1]{inputenc}
\usepackage[T1]{fontenc}
\usepackage[brazil]{babel}
\usepackage[alf]{abntcite}
\usepackage[ordem=alf]{tabela-simbolos}
\usepackage{url}
\usepackage{graphicx}
\usepackage{listings}
\usepackage{verbatim}
\usepackage{subfigure}
\usepackage{multicol}
\usepackage{framed}
% O comando abaixo define o diretorio onde devem ser colocadas as imagens.
% Neste caso o diretorio e' ./imagens/
\graphicspath{ {./imagens/} }
\def\lstlistingname{Listagem}

%%%%%%%%%%%%%%%%%%%%%%%%%%%%%%%%%%%%%%%%%%%%%%%%%%%%
% Dados da monografia
%%%%%%%%%%%%%%%%%%%%%%%%%%%%%%%%%%%%%%%%%%%%%%%%%%%%

\newcommand{\meunome}{Fulano de tal}
\newcommand{\meutitulo}{Um modelo LaTeX para monografias no DCC-UFBA}
\newcommand{\meusubtitulo}{Uma abordagem pr�tica}
\newcommand{\meuano}{2013}
\newcommand{\meuorientador}{Orientadora: \profa\ Christina von Flach Garcia
Chavez}

%%%%%%%%%%%%%%%%%%%%%%%%%%%%%%%%%%%%%%%%%%%%%%%%%%%%

%% O comando \obs aqui definido permite que o autor faca anotacoes na
%% monografia que aparecem no PDF gerado. Para ativar o comando, descomente
%% a primeira linha e comente a segunda.
%% Exemplo de uso: \obs{Preciso melhorar este par�grafo...}

%\newcommand{\obs}[1]{\underline{\textbf{OBSERVA��O}}: \emph{#1}}
\newcommand{\obs}[1]{}

\def\ordfem{\mbox{\raise .35em \hbox{\underline{\scriptsize a}\ }}}
\def\ordmasc{\mbox{\raise .35em \hbox{\underline{\scriptsize o}\ }}}
\def\profa{Prof\ordfem.}

%%%%%%%%%%%%%%%%%%%%%%%%%%%%%%%%%%%%%%%%%%%%%%%%%%%%

\begin{document}

\input{capa.tex}
\input{folhaderosto.tex}

\begin{resumo}
Nonono nonono nonono, nonono, nonono nonono nonono nononono nonno. 
Nonono nonono nonono, nonono, nonono nonono nonono nononono nonno. 
Nonono nonono nonono, nonono, nonono nonono nonono nononono nonno. 
Nonono nonono nonono, nonono, nonono nonono nonono nononono nonno. 
Nonono nonono nonono, nonono, nonono nonono nonono nononono nonno. 
Nonono nonono nonono, nonono, nonono nonono nonono nononono nonno. 
Nonono nonono nonono, nonono, nonono nonono nonono nononono nonno. 

\textbf{Palavras-chave:}
monografia,
gradua��o, 
projeto final.
\end{resumo}

% O abstract e' opcional.
\begin{abstract}
Nonono nonono nonono, nonono, nonono nonono nonono nononono nonno. 
Nonono nonono nonono, nonono, nonono nonono nonono nononono nonno. 
Nonono nonono nonono, nonono, nonono nonono nonono nononono nonno. 
Nonono nonono nonono, nonono, nonono nonono nonono nononono nonno. 
Nonono nonono nonono, nonono, nonono nonono nonono nononono nonno. 
Nonono nonono nonono, nonono, nonono nonono nonono nononono nonno. 
Nonono nonono nonono, nonono, nonono nonono nonono nononono nonno. 

\textbf{Keywords:} 
monograph,
graduation,
final project.
\end{abstract}

%% As listas a seguir sao opcionais:
\listadefiguras
%\listadetabelas
\listadesiglas
%\listadesimbolos

\sumario

% O conteudo da monografia esta' nos seguintes arquivos:
\chapter{Introdu��o}

Modelo de monografia usando as normas ABNT (Associa��o Brasileira de Normas
T�cnicas) \sigla{ABNT}{Associa��o Brasileira de Normas T�cnicas} e adapta��o
personalizada do padr�o do Departamento de Ci�ncia da Computa��o (DCC) da
Universidade Federal da Bahia (UFBA).
\sigla{DCC}{Departamento de Ci�ncia da Computa��o}
\sigla{UFBA}{Universidade Federal da Bahia}
Fontes latex cedidos pela ABNT e disponibilizados por 
Maur�cio Vieira. (Valeu Maurix!). Adaptado por Abelmon Bastos por solicita��o
da \profa\ D�bora Abdalla para o semestre 2005.1.
Adaptado por Rodrigo Rocha por solicita��o da \profa\ D�bora Abdalla no fim
do semestre 2007.1.
Atualizado por Ibirisol em 2013.1 para adequa��o ao \cite{manualVisual} da UFBA.

% use sua propria estrutura

%Problema etc

%Objetivo etc

%Resultados esperados etc

%Estrutura/Organiza��o etc

\input{redesemfioemmalha.tex}
\input{desenvolvimento.tex}
\input{conclusao.tex}
\input{apendices.tex}

\bibliography{monografia}

\end{document}
